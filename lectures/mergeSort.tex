\section{Сортировка слиянием | Merge Sort}

\subsection{Разделяй и властвуй}

Алгоритм использует принцип «разделяй и властвуй»: задача разбивается на подзадачи меньшего размера, которые решаются по отдельности, после чего их решения комбинируются для получения решения исходной задачи. Конкретно процедуру сортировки слиянием можно описать следующим образом:

\begin{enumerate}
    \item Если в рассматриваемом массиве один элемент, то он уже отсортирован — алгоритм завершает работу.
    \item Иначе массив разбивается на две части, которые сортируются рекурсивно.
    \item После сортировки двух частей массива к ним применяется процедура слияния, которая по двум отсортированным частям получает исходный отсортированный массив.

\end{enumerate}

\subsection{Слияние двух массивов}

У нас есть два массива a
и b
(фактически это будут две части одного массива, но для удобства будем писать, что у нас просто два массива). Нам надо получить массив c
размером |a|+|b|
. Для этого можно применить процедуру слияния. Эта процедура заключается в том, что мы сравниваем элементы массивов (начиная с начала) и меньший из них записываем в финальный. И затем, в массиве у которого оказался меньший элемент, переходим к следующему элементу и сравниваем теперь его. В конце, если один из массивов закончился, мы просто дописываем в финальный другой массив. После мы наш финальный массив записываем заместо двух исходных и получаем отсортированный участок.

\subsection{Псевдокод процедуры слияния}

Ниже приведён псевдокод процедуры слияния, который сливает две части массива a
— [left;mid)
и [mid;right)

\begin{algorithmic}[1]
    \Function{merge}{a: \textbf{int}[], left, mid, right: \textbf{int}}
        \State it1 $\gets$ 0
        \State it2 $\gets$ 0
        \State result: \textbf{int}[right - left]
        \While{left + it1 $<$ mid \textbf{and} mid + it2 $<$ right}
            \If{a[left + it1] $<$ a[mid + it2]}
                \State result[it1 + it2] $\gets$ a[left + it1]
                \State it1 $\gets$ it1 + 1
            \Else
                \State result[it1 + it2] $\gets$ a[mid + it2]
                \State it2 $\gets$ it2 + 1
            \EndIf
        \EndWhile
        \While{left + it1 $<$ mid}
            \State result[it1 + it2] $\gets$ a[left + it1]
            \State it1 $\gets$ it1 + 1
        \EndWhile
        \While{mid + it2 $<$ right}
            \State result[it1 + it2] $\gets$ a[mid + it2]
            \State it2 $\gets$ it2 + 1
        \EndWhile
        \For{i = 0 \textbf{to} it1 + it2}
            \State a[left + i] $\gets$ result[i]
        \EndFor
    \EndFunction
\end{algorithmic}

\subsection{Рекурсивная реализация}
Рекурсивная реализация сортировки слиянием заключается в разбинии массива на две части и сортировке этих частей.
Таким образом, массив обе части массива сортируются, а потом сливаются в один процедурой $merge$.\\
Приведенный ниже псевдокод сортирует полуинтервал массива $[left; right)$
\\
\begin{algorithmic}[1]
    \Function{mergeSortRecursive}{a : \textit{int[n]}, left, right : \textit{int}}
        \If{$\textit{left} + 1 \geq \textit{right}$}
            \State \Return
        \EndIf
        \State $\textit{mid} = \frac{\textit{left} + \textit{right}}{2}$
        \State \Call{mergeSortRecursive}{a, left, mid}
        \State \Call{mergeSortRecursive}{a, mid, right}
        \State \Call{merge}{a, left, mid, right}
    \EndFunction
\end{algorithmic}


\section{Итеративная реализация}


\begin{algorithmic}[1]
    \Function{mergeSortIterative}{a : \textit{int[n]}}
        \For{$i = 1$ \textbf{to} $n$, $i *= 2$}
            \For{$j = 0$ \textbf{to} $n - i$, $j += 2 * i$}
                \State \Call{merge}{a, j, j + i, \text{min}(j + 2 * i, n)}
            \EndFor
        \EndFor
    \EndFunction
\end{algorithmic}


Исходный массив делится на подмассивы размером 1,2,4,8 и так далее по степеням двойки.
Далее для каждого такого подмассива происходит слияние двух подмассивов:
\begin{enumerate}
    \item Первый подмассив начинается с индекса $j$ и имеет длину $i$.
    \item  Второй подмассив начинается с индекса $j + i$ и имеет длину $i$, но не выходит за пределы массива $a$.
    \item $min(j + 2 * i, n)$ гарантирует, что конец второго подмассива не превышает длину массива $a$.
\end{enumerate}

Данный алгоритм использует итеративный подход для сортировки массива слиянием, проходя через массив с подмассивами увеличивающейся длины и сливая их поэтапно. Такой подход позволяет избежать затрат памяти на рекурсивные вызовы, делая алгоритм более эффективным с точки зрения использования памяти.

\subsection{Асимптотика}

\begin{theorem}[Асимптотика процедуры слияния по памяти]

    Для хранения массива результата слияния массивов размерами $n_1 + n_2 = n$ займёт $O(n_1+n_2)=O(n)$ памяти.

\end{theorem}

\begin{theorem}[Асимптотика процедуры слияния по времени]

    Итеративный проход по двум массивам размерами $n_1 + n_2 = n$ займёт $O(n_1+n_2)=O(n)$ времени.

\end{theorem}



\begin{theorem}[Асимптотика процедуры слияния по времени]

    Итеративный проход по двум массивам размерами $n_1 + n_2 = n$ займёт $O(n_1+n_2)=O(n)$ времени.

\end{theorem}


\begin{theorem}[Временная сложность сортировки слиянием]

    Время работы сортировки слиянием массива размера $n$ равно $O(nlog\,{n})$

\end{theorem}

\begin{proof}
    Составим рекуррентное соотношение \[T(n) = 2\,T(\frac{n}{2})+O(n)\], где $T(n)$ — время сортировки массива размером $n$, $T(n/2)$
    — время сортировки массива размером $\frac{n}{2}$, $O(n)$ — время слияния двух массивов размера $\frac{n}{2}$.
    Тогда время сортировки равно:

    \[ T(n) = 2T\left(\frac{n}{2}\right) + O(n) = 4T\left(\frac{n}{4}\right) + 2O(n) = \cdots = T(1) + \log(n)O(n) = O(n \log(n)). \]
\end{proof}

\begin{theorem}[Пространственная сложность сортировки слиянием]

    Пространственная сложность работы сортировки слиянием массива размера $n$ равно $O(n)$

\end{theorem}

\begin{proof}
    Рассмотрим последний шаг сортировки слиянием:\\
    имеется два отсортированных массива размера $\frac{n}{2}$ и исходный массив, в который мы сольем итоговый отсортированный массив.
    Так как мы храним данной итерации два дополнительных массива размера $\frac{n}{2}$, то сложность будет равна $O(\frac{n}{2} + \frac{n}{2})=O(n)$

    В рекурсивной реализации сортировки слияением также будут затраты $O(log\,{n})$ на хранение стека рекурсии, но на асимптотику это не повлияет, так как \[O(n + log\,{n})=O(n)\]
\end{proof}

\subsection{Примечания}
Сортировка слиянием является устойчивой.